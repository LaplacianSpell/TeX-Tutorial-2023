%%%%%%%%%%%%%%%%%%%%%%%%%%%%%%%%
%  LaplacianSpell (c)          %
%  pd20@mails.tsinghua.edu.cn  %
%  限清华大学内部使用(不是)  %
%%%%%%%%%%%%%%%%%%%%%%%%%%%%%%%%

%% 导言区

% 标记文章的类别
\documentclass[12pt, letterpaper]{article}

% 常见宏包的使用,zyf学长介绍
\usepackage[utf8]{inputenc}
\usepackage{amsmath}
\usepackage{graphicx} % 需要使用这个包
\usepackage{float}
\usepackage[UTF8]{ctex}

% 作者等信息
\title{LaTeX}
\author{Author}
\date{March 2023} %\today

%%%%我%%是%%分%%割%%线%%%%

%% 主体区

% 文档开始声明
\begin{document}

\maketitle

% 摘要体
\begin{abstract}
This is a simple paragraph at the beginning of the 
document. A brief introduction about the main subject.
\end{abstract}

% 添加目录
\tableofcontents

% 新的一页
\newpage

%%%%%%%%%%%%%%%%%%%%%%%%%%%

% 节,演示空格缩进等
\section{section on basic structure}

This is a section, I added some s\ p\ a\ c\ e.

And there is an indent.

\noindent And I cancelled it.

And the new \\ line

how about new page?

\newpage


%%%%%%%%%%%%%%%%%%%%%%%%%%%

% 演示次级分节,段落等

% 全部不缩进了
\setlength{\parindent}{0pt}

\subsection{subsection}

Yes?

% 没有缩进
No!

\subsubsection{subsubsection}

%保留字
 \# \$ \% \^{} \& \_ \{ \} \~{}

\paragraph{paragraph}

\subparagraph{subparagraph}

\newpage
%%%%%%%%%%%%%%%%%%%%%%%%%%%

% 写公式
inline $\partial_\mu A_\nu-\partial_\nu A_\mu$ form

single line
$$
\frac{\partial^2}{\partial t^2}
$$
form

another single line

\[
\frac{\partial^2}{\partial t^2}
\]

another single line

\begin{equation}
    \frac{\partial^2}{\partial t^2}
\end{equation}

$\alpha\beta\gamma\Omega\omega\epsilon\varepsilon\pi\varpi$

$h^\prime$

$\hbar\bar{h}$

$\tilde{a}$

$\bar{s}$

$$\int\oint$$

$$2^{34567} $$
$$^3He$$  
$$D_3$$
$$^4_2He$$

Taylor

\begin{equation}
f(x) = \sum_{i=0}^{\infty} \frac{f^{(i)}(x_0)(x-x_0)^i}{i!}
\end{equation}

$$
\Sigma_o^\infty
$$

\newpage

%%%%%%%%%%%%%%%%%%%%%%%%%%%

% 演示字体


\textit{words in italics}

\textsl{words slanted}

\textsc{words in smallcaps}

\textbf{words in bold}

\texttt{words in teletype}

\textsf{sans serif words}

\textrm{roman words}

\underline{underlined words}

\newpage
%%%%%%%%%%%%%%%%%%%%%%%%%%%

% 表格

%% 无序表格
\begin{itemize}
  \item[-] The individual entries are indicated with a black dot, a so-called bullet.
  \item The text in the entries may be of any length.
\end{itemize}


% 有序表格(使用[]对标记进行替换)
\begin{enumerate}
  \item This is the first entry in our list.
  \item[-] The list numbers increase with each entry we add.
\end{enumerate}

% 没有边框表格
\begin{center}

% 这{c c c}里面的这几个告诉的是列信息
% c表示该列居中对齐,l,r分别表示左对齐和右对齐
\begin{tabular}{l c c} 

 cell1 & cell2 & cell3 \\ % &用来分割列,\\用来换行
 cell4 & cell5 & cell6 \\  
 cell7 & cell8 & cell9    
\end{tabular}

\end{center}

% 有边框
% 如果是 |c|c|c| 则会在列之间绘制竖线.
\begin{center}
\begin{tabular}{|c|c|c|} 
 \hline % 表示在这里插入一个贯穿所有列的横着的分割线
 cell1 & cell2 & cell3 \\ 

\hline % 会在第一列和第二列的竖线之间插入一个横着的分割线。
 cell4 & cell5 & cell6 \\ 
 cell7 & cell8 & cell9 \\ 
 \hline % 表示在这里插入一个贯穿所有列的横着的分割线;
\end{tabular}
\end{center}

% 矩阵

% 注意使用包 \usepackage{amsmath}

$\begin{matrix}
a & b \\
c & d \\
\end{matrix}$

$\begin{pmatrix}
a & b \\
c & d \\
\end{pmatrix}$

$\begin{bmatrix}
a & b \\
c & d \\
\end{bmatrix}$

$\begin{vmatrix}
a & b \\
c & d \\
\end{vmatrix}$

%%%%%%%%%%%%%%%%%%%%%%%%%
% 添加图片


% \usepackage{graphicx} % 需要使用这个包
% \usepackage{float}

\begin{figure}[H] % 这个参数是告诉编译器把图片放什么位置的,我这里是告诉他放在原地.
  \centering % 使用命令使得图片居中对齐
  \includegraphics[scale=0.5]{graph/logo.jpg} % 同一文档之下的图片,或者可以使用参数height=0.4 weight=0.3
  \caption{THUPSAST} % 图的题目
\end{figure}.


\end{document}
